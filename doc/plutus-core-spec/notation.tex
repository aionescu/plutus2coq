\section{Some basic notation}
We begin with some notation which will be used throughout the document.

\subsection{Sets}
\label{sec:notation-sets}
\begin{itemize}
  \item The symbol $\disj$ denotes a disjoint union of sets;  for emphasis we often use this
    to denote the union of sets which we know to be disjoint.
  \item Given a set $X$, $X^*$ denotes the set of finite sequences of elements of $X$:
    $$
    X^*= \bigdisj{\{X^n: n \in \mathbb{N}\}}.
    $$
  \item $\N = \{0,1,2,3,\ldots\}$.
  \item $\Nplus = \{1,2,3,\ldots\}$.
  \item $\Nab{a}{b} = \{n \in \N: a \leq n \leq b\}$.
  \item $\B = \Nab{0}{255}$, the set of 8-bit bytes.
  \item $\B^*$ is the set of all bytestrings.
  \item $\Z = \{\ldots, -2, -1, 0, 1, 2, \ldots\}$.
  \item $\mathbb{F}_q$ denotes a finite field with $q$ elements ($q$ a prime power).
  \item $\mathbb{F}_q^*$ denotes the multiplicative group of nonzero elements of $\mathbb{F}_q$.
  \item $\U$ denotes the set of Unicode scalar values, as defined in~\cite[Definition D76]{Unicode-standard}.
  \item $\U^*$ is the set of all Unicode strings.
  \item We assume that there is a special symbol $\errorX$ which does not appear
    in any other set we mention.  The symbol $\errorX$ is used to indicate that
    some sort of error condition has occurred, and we will often need to consider
    situations in which a value is either $\errorX$ or a member of some set $S$.
    For brevity, if $S$ is a set then we define
    $$
    \withError{S} := S \disj \{\errorX\}.
    $$
\end{itemize}%
\nomenclature[An1]{$\N$}{$\{0,1,2,3,\ldots\}$}%
\nomenclature[An2]{$\Nplus$}{$\{1,2,3,\ldots\}$}%
\nomenclature[An3]{$\Nab{a}{b}$}{$\{n \in \N: a \leq n \leq b\}$}%
\nomenclature[Aw]{$\Z$}{$\{\ldots, -2, -1, 0, 1, 2, \ldots\}$}%
\nomenclature[Af]{$\mathbb{F}_q$}{The finite field with $q$ elements}%
\nomenclature[Af]{$\mathbb{F}_q^*$}{The multiplicative group of $\mathbb{F}_q$}%
\nomenclature[Ab]{$\B$}{$\{n \in \Z: 0 \leq n \leq 255\}$}%
\nomenclature[Ab]{$\B^*$}{The set of all bytestrings}%
\nomenclature[Au]{$\U$}{The set of Unicode scalar values}%
\nomenclature[Au]{$\U^*$}{The set of Unicode strings}%
\nomenclature[Ax]{$\withError{S}$}{$S \disj \{\errorX\}$ ($S$ a set)}%
\nomenclature[Azz]{$\disj$}{Disjoint union of sets}%
\nomenclature[Azz]{$X^*$}{The set of all finite sequences of elements of a set $X$}

\subsection{Lists}
\label{sec:notation-lists}
\begin{itemize}
\item  The symbol $[]$ denotes an empty list.
\item The notation $[x_m, \ldots, x_n]$ denotes a list containing the elements
  $x_m, \ldots, x_n$.  If $m>n$ then the list is empty.
\item The length of a list $L$ is denoted by $\length(L)$.
\item Given two lists $L = [x_1,\ldots, x_m]$ and $L' = [y_1,\ldots, y_n]$, $L\cdot L'$ 
denotes their concatenation  $[x_1,\ldots, x_m,$ $y_1, \ldots, y_n]$.  % Broken in the middle to keep it out of the margin.
\item Given an object $x$ and a list $L = [x_1,\ldots, x_n]$,
we denote the list $[x,x_1,\ldots, x_n]$ by $x \cons L$.
\item Given a list $L = [x_1, \ldots, x_n]$ and an object $x$,
we denote the list $[x_1, \ldots, x_n, x]$ by $L \snoc x$.
%%\item We say that the list $L'$ is a \textit{proper prefix} of the list
%%  $L = [x_1, \ldots, x_n]$, and
%%  write $L' \prec L$,  if $L' = [x_1, \ldots, x_m]$ for some $m<n$.
\item In the special case of bitstrings (ie, lists of elements of $\{0,1\}$) we
  sometimes use notation such as \texttt{101110} to denote the list
  $[1,0,1,1,1,0]$; we use a teletype font to avoid confusion with decimal
  numbers.  
\item Given a syntactic category $V$, the symbol $\repetition{V}$ denotes a
  possibly empty list $[V_1,\ldots, V_n]$ of elements $V_i \in V$.
\end{itemize}%
\nomenclature[B1]{$[]$}{The empty list}%
\nomenclature[B2]{$L \cdot L'$}{Concatenation of lists $L$ and $L'$}%
\nomenclature[B3]{$x \cdot L$}{$[x] \cdot L$}%
\nomenclature[B4]{$L \cdot x$}{$L \cdot [x]$}%
\nomenclature[B5]{$\repetition{V}$}{A sequence $[V_1,\ldots, V_n]$}%
\nomenclature[B6]{$\length(\cdot)$}{Length of a list or bytestring}

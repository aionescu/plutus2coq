\section{Introduction}

If we look at contracts on Ethereum in terms of their use and the monetary values that they process, then it becomes apparent that so-called \emph{user-defined} (or \emph{custom}) \emph{tokens} play a central role in that ecosystem. The two most common token types are \emph{fungible tokens}, following the ERC-20 standard~\cite{ERC-20}, and \emph{non-fungible} tokens, following the ERC-721 standard~\cite{ERC-721}.
% This footnote is IMHO superflous. The dictionary definition of "fungible" is " replaceable by another identical item; mutually interchangeable", which seems to perfectly fit with our use and we use the term as it is usually used in the context of cryptocurrencies. (This footnote also screws up the frontpage.)
%\footnote{
%  Fungibility is strictly a relation between objects, rather than a property of a single object. However, throughout this paper we will follow the common pattern of describing an object as (non-)fungible when it is (non-)fungible with all other objects in some appropriate reference class, which should be clear from context (e.g. a ``fungible token'' is fungible with all the other tokens in its asset group, but not outside that). We may also refer to a set of objects as (non-)fungible when they are all (non-)fungible with each other.
%}

On Ethereum, ERC-20 and ERC-721 tokens are fundamentally different from the native cryptocurrency, Ether, in that their creation and use always involves user-defined custom code --- they are not directly supported by the underlying ledger, and hence are \emph{non-native}.  This makes them unnecessarily inefficient, expensive, and complex. Although the ledger already includes facilities to manage and maintain a currency, this functionality is replicated in interpreted user-level code, which is inherently less efficient. Moreover, the execution of user-code needs to be paid for (in gas), which leads to significant costs. Finally, the ERC-20 and ERC-721 token code gets replicated and adapted, instead of being part of the system, which complicates the creation and use of tokens and leaves room for human error.

The alternative to user-level token code is a ledger that supports \emph{native tokens}. In other words, a ledger that directly supports
\begin{inparaenum}[(1)]
\item the creation of new user-defined token or asset types,
\item the forging of those tokens, and
\item the transfer of custom token ownership between multiple participants.
\end{inparaenum}
In a companion paper~\cite{plain-multicurrency}, we propose a generalisation of Bitcoin-style \UTXO\ ledgers, which we call \UTXOma (``ma'' for ``multi-asset''), adding native tokens by way of so-called \emph{token bundles} and domain-specific \emph{forging policy scripts}, without the need for a general-purpose scripting system.

Independently, we previously introduced the \emph{Extended UTXO Model} (\EUTXO)~\cite{eutxo-1-paper} as an orthogonal generalisation of Bitcoin-style \UTXO\ ledgers, enabling support of more expressive smart contracts, with functionality similar to contracts on Ethereum. To support user-defined tokens or currencies on \EUTXO, we could follow Ethereum's path and define standards corresponding to ERC-20 and ERC-721 for fungible and non-fungible tokens, but then we would be subject to the same disadvantages that non-native tokens have on Ethereum.

In this paper, to avoid the disadvantages of non-native tokens, we investigate the combination of the two previously mentioned extensions of the plain \UTXO\ model: we add \UTXOma-style token bundles and asset policy scripts to the \EUTXO\ model, resulting in a new \EUTXOma\ ledger model.

We will show that the resulting \EUTXOma\ model is strictly more expressive than both \EUTXO\ and \UTXOma\ by itself. In particular, the constraint-emitting state machines in~\cite{eutxo-1-paper} can not ensure that they are initialised correctly. In \EUTXOma, we are able to use non-fungible tokens to trace \emph{threads} of state machines. Extending the mechanised Agda model from~\cite{eutxo-1-paper} allows us to then prove inductive and temporal properties of state machines by induction over their traces, covering a wide variety of state machine correctness properties.

Moreover, the more expressive scripting functionality and state threading of \EUTXO\ enables us to define more sophisticated asset policies than in \UTXOma. Additionally, we argue that the combined system allows for sophisticated access-control schemes by representing roles and capabilities in the form of user-defined tokens.

In summary, this paper makes the following contributions:
%
\begin{itemize}
\item We introduce the multi-asset \EUTXOma\ ledger model (\S\ref{sec:EUTXOma}).
\item We outline a range of application scenarios that are arguably better supported by the new model, and also applications that plain \UTXOma\ does not support at all (\S\ref{sec:applications}).
\item We formally prove a transfer result for inductive and temporal properties from constraint emitting machines to the \EUTXOma\ ledger, an important property that we were not able to establish for plain \EUTXO\ (\S\ref{sec:formalization}).
\end{itemize}
%
We discuss related work in \S\ref{sec:related-work}. Due to space constraints, the formal ledger rules for \EUTXOma\ are in Appendix~\ref{app:model}. A mechanised version of the ledger rules and the various formal results from \S\ref{sec:formalization} is available as Agda source code.\footnote{\url{\agdaRepo}}

On top of the conceptual and theoretical contributions made in this paper, we would like to emphasise that the proposed system is highly practical. In fact, \EUTXOma\ underlies our implementation of \emph{Plutus Platform,} the smart contract system of the Cardano blockchain.\footnote{\url{https://github.com/input-output-hk/plutus}}
